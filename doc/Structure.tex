\documentclass[a4paper]{article}

\usepackage{hyperref}
\usepackage{amsmath}
\usepackage{empheq}
\usepackage{mathtools}
\usepackage[german]{babel}
\usepackage[utf8]{inputenc}
\usepackage{amsfonts}
\usepackage{graphicx}
\usepackage{siunitx}
\usepackage{tabularx}
\usepackage{wrapfig}
\usepackage{float}
\usepackage{titlesec}
\usepackage[margin=2cm]{geometry}


%opening
\title{}
\author{}
\title{smartHome}
\author{Til Blechschmidt\\ Student, Otto-Hahn-Gymnasium Geesthacht, Germany}


\begin{document}

	\pagenumbering{gobble}
	\maketitle
	\newpage
	\pagenumbering{arabic}
	
	\tableofcontents
	\listoffigures
	\listoftables
	\newpage
	
	%\setcounter{secnumdepth}{4}
	
%	\titleformat{\paragraph}
%	{\normalfont\normalsize\bfseries}{\theparagraph}{1em}{}
%	\titlespacing*{\paragraph}
%	{0pt}{3.25ex plus 1ex minus .2ex}{1.5ex plus .2ex}

    \section{Philosophy}
        This project aims to be a decentralized SmartHome protocol that is capable of being used for a wide variety of
        devices without much hassle and targets simple setup and usage as one of the main goals.
	\section{Channels}
	    A channel describes a group or collection of Devices. A channel by itself is nothing more than a ID in the
	    heap of some devices. It can be referenced to by a string defined by the user such as 'Living Room' or 'Kitchen'
	    by utilizing the Channel DNS stack. A group of devices listen to requests which are aimed at a specific channel
	    and reacts to possibly but not exclusively only those.
	\section{Devices}
	    \subsection{Passive devices}
            A passive device consists of two parts. The software layer and the hardware layer both in conjunction
            enabling the network to interact with devices in the real world like lamps or TVs.
            \paragraph{Software layer} The software layer responds to requests coming in from the network. The responses or
                actions might be limited to a specific channel but can use various wild-cards as well. That's up to the
                developer and use-case. Note that a software layer is NOT limited to representing one device. It is possible
                for a software layer to run multiple instances of class that reacts to incoming messages. Therefore it is
                possible for one software layer to control multiple hardware layers like lamps for example.

            \paragraph{Hardware layer} The hardware layer represents the part of the program that takes commands from the
                software layer and translates them to either interactable items like a screen or trigger some actions in
                the real world like a switch to turn on a lamp.
        \subsection{Active devices}
            A active device is sending status updates and requests to other passive devices reacting to those. An
            example for an active device would be a smart-phone. It can enter a specific perimeter or receive a call and
            sends these changes in state to the network. Note however that these changes are not saved by any
            central authority. All devices that would react to the change in state do so upon reception of the datagram
            and do not store the state by itself. But its possible for any 3rd party to add a device that stores these
            states for later access like showing them to a user that was disconnected from the network when the change
            in state took place.
	\section{Channel DNS}
	    Since humans usually can't follow along with huge amounts of numbers the channel IDs used in the Communication
	    protocol are translated to strings by a Channel DNS system. In its most basic form this system passes around
	    changes that were done to the registry including a timestamp. Every device in the network receives a
	    copy of this registry and by adding up all changes it is able to resolve channel names to IDs.
	    \paragraph{Fetching full registry} Since devices which just joined the network don't have the initial stack and
	        would have to wait for a change to appear (which in turn would require the whole registry to be sent to
	        every device) a device that joins a network asks for a up-to-date registry upon connection. This
	        includes the problem that all devices would respond with the same data, creating unnecessary overhead. In
	        order to prevent this problem the newly connected device asks waits for any abitrary datagram to be
	        transmitted across the multicast channel and then asks the sender of that datagram for the registry.
        \paragraph{Incremental updates}  Because all devices have synced their registries on their first connect it is
            only required to send incremental updates. So if any device is requesting a change to a registry entry it
            broadcasts this change together with a timestamp to all connected devices which in turn insert this change
            into their registry.
        \paragraph{Registry overflow} As all changes are written into the registry rather than the current state it will
            overflow at some point. To counteract this problem all devices resolve their registries on receiving an
            update and delete all unnecessary entries like multiple consecutive renames or in case of a deletion all
            previous changes to that entry.
    \section{Communication protocols}
        \subsection{UDP}
	        \subsubsection{General structure}
		        All UDP communication is split into two parts. One being direct communication between two devices
		        following a special protocol depending on the task and the other part being indirect communication via
		        multicast sockets.
		    \subsubsection{Multicast}
			    The datagrams in the multicast channel are all JSON formatted. This JSON object may have the following
			    keys:\\
			    \begin{tabular}{l | l}
			    	type & uint\_8\\
			    	action & bool\\
			    	payload & object
			    \end{tabular}
	        \subsubsection{Device scanning}
            \subsubsection{Updates}
            \subsubsection{Read \& Write workflow}
            \subsubsection{Channel DNS incremental synchronisation}
        \subsection{TCP}
            \subsubsection{Channel DNS registry synchronisation}
\end{document}
